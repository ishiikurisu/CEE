\documentclass[12pt, a4paper, twoside]{article}
\usepackage[utf8]{inputenc}
\usepackage[cm]{fullpage}
\usepackage{fancyhdr}
\usepackage{textcomp}
\usepackage{graphicx}
\usepackage{commath}
\usepackage[portuguese]{babel}

\begin{document}

\title{Modelagem teórica de um gerador elétrico para instalações pluviais}
\author{Cristiano Silva Júnior: 13/0070629 \\
Rafael Costa: 13/0144908}
\date{\today}
\maketitle

\section{Introdução}

Diante de evidentes problemas ambientais relacionados à produção indevida de
energia elétrica [1], se faz necessário procurar novas formas de gerar e
distribuir energia elétrica sem causar danos ao meio ambiente. Formas mais
comuns encontram-se nas gerações eólica e solar, onde o ser humano aproveita
efeitos naturais relativamente infinitos para extrair sua subsistência. Nesse
sentido, a geração hidroelétrica também poderia ser utilizada.

Apesar das usinas hidroelétricas ainda requererem uma construção e uma
manutenção de altos custos [2], outras dinâmicas envolvendo fluídos poderiam
ser reaproveitadas. Neste sentido, é possível utilizar, por exemplo, a energia
disponível nos fluídos em encanamentos para acionar geradores elétricos capazes
de auxiliar na produção de energia elétrica e aliviar a rede elétrica
convencional.

Este artigo propõem um modelo de instalação de um gerador elétrico que converte
o movimento de fluídos em encanamentos de redes de esgoto em energia elétrica.
Para tanto, avaliamos o que já existe em termos de instalações pluviais ao
redor do mundo para podermos dimensionar possíveis máquinas elétricas
aplicáveis à proposta.

\section{Metodologia}

\subsection{Redes pluviais}

Suponha um encanamento preenchido com água sendo bombeada a uma vazão $A$ e a uma pressão $D$. A partir disto, o objetivo deste projeto é avaliar se um gerador acoplado a este encanamento consegue extrair uma potência mecânica razoável o suficiente para ser utilizado comercialmente.

% TODO Adicionar uma figura que exemplique o procedimento a seguir.

Por definição, a potência transferida $P$ ao gerador será proporcional ao torque $T$ gerado e à velocidade angular $\omega$. Sendo assim, precisamos descobrir qual o torque e qual a velocidade angular que o encanamento com os parâmetros fornecidos consegue fornecer ao gerador.

A fim de simplificar a análise, vamos supor que o regime de ?? é uniforme e, portanto, que o perfil de velocidade do fluído dentro do cano obedece uma lei quadrática. Consideraremos a geometria do rotor do gerador como sendo uma similar à do projeto X [4]. Como parâmetros construtivos, consideraremos que a turbina do gerador possui um raio $ r $ e que ele possui uma eficiência $ \eta $.

Utilizando a definição de torque e de pressão, é fácil ver que $$ T = F A r $$ Analogamente, podemos utilizar a definição de velocidade angular para ver que $$ \omega = V(r) r $$ Portanto, a potência será $$ P = F A V(r) r^2 $$

\subsection{Máquinas elétricas}

% TODO Descobrir qual a máquina elétrica que é cabível aqui.

\section{Resultados}

Para obter os resultados, utilizamos como referência a Estação de Tratamento de Água Vale do Amanhecer, que abastece a região de mesmo nome. A vazão média de 29 L/s foi obtida no site da CAESB [3] e a altitude da estação em relação à região abastecida, de aproximadamente 40 metros, foi obtida utilizando um mapa topográfico.



\section{Conclusão}

Para escrever a conclusão, ainda precisamos fazer contas para saber o que é
possível ou não.

\section{Referência Bibliográfica}

\begin{enumerate}
    \item Problemas ambientais relacionados à geração tradicional de energia
    elétrica
    \item Custos na geração de energia elétrica
    \item http://atlascaesb.maps.arcgis.com/apps/MapJournal/index.html?appid=4d06131962ca482a9d51502c630e195f
    \item Geometria do gerador (Lucid Energy?)
\end{enumerate}

\end{document}
