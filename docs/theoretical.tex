\documentclass[12pt, a4paper, twoside]{article}
\usepackage[utf8]{inputenc}
\usepackage[cm]{fullpage}
\usepackage{fancyhdr}
\usepackage{textcomp}
\usepackage{graphicx}
\usepackage{commath}
\usepackage[portuguese]{babel}

\begin{document}

\title{Modelagem teórica de um gerador elétrico para instalações pluviais}
\author{Cristiano Silva Júnior: 13/0070629}
\date{\today}
\maketitle

\section{Introdução}

Diante de evidentes problemas ambientais relacionados à produção indevida de
energia elétrica [1], se faz necessário procurar novas formas de gerar e
distribuir energia elétrica sem causar danos ao meio ambiente. Formas mais
comuns encontram-se nas gerações eólica e solar, onde o ser humano aproveita
efeitos naturais relativamente infinitos para extrair sua subsistência. Nesse
sentido, a geração hidroelétrica também poderia ser utilizada.

Apesar das usinas hidroelétricas ainda requererem uma construção e uma
manutenção de altos custos [2], outras dinâmicas envolvendo fluídos poderiam
ser reaproveitadas. Neste sentido, é possível utilizar, por exemplo, a energia
disponível nos fluídos em encanamentos para acionar geradores elétricos capazes
de auxiliar na produção de energia elétrica e aliviar a rede elétrica
convencional.

Este artigo propõem um modelo de instalação de um gerador elétrico que converte
o movimento de fluídos em encanamentos de redes de esgoto em energia elétrica.
Para tanto, avaliamos o que já existe em termos de instalações pluviais ao
redor do mundo para podermos dimensionar possíveis máquinas elétricas
aplicáveis à proposta.

\section{Metodologia}

\subsection{Redes pluviais}

% TODO Descrever os parâmetros medíveis de uma rede de esgoto.

% TODO Descrever a rede pluvial de Brasília.

% TODO Descrever a rede pluvial de uma cidade do interior.

\subsection{Máquinas elétricas}

% TODO Descobrir qual a máquina elétrica que é cabível aqui.

\section{Conclusão}

Para escrever a conclusão, ainda preciso fazer contas para saber o que é
possível ou não.

\section{Referência Bibliográfica}

\begin{enumerate}
    \item Problemas ambientais relacionados à geração tradicional de energia
    elétrica
    \item Custos na geração de energia elétrica
\end{enumerate}

\end{document}
