\documentclass[12pt, a4paper, twoside]{article}
\usepackage[utf8]{inputenc}
\usepackage[cm]{fullpage}
\usepackage{fancyhdr}
\usepackage{textcomp}
\usepackage{graphicx}
\usepackage{commath}
\usepackage[portuguese]{babel}

\begin{document}

\title{Presença das Máquinas Elétricas na Sociedade}
\author{Cristiano Silva Júnior: 13/0070629}
\date{\today}
\maketitle

Desde o desenvolvimento das leis de Faraday no século XIX até os dias atuais,
a humanidade viveu mudanças profundas em seu cotidiano diante das diversas
transformações que o uso da energia elétrica e da magnética foram capazes de
trazer. Boa parte dessas mudanças se devem à implementação de máquinas
elétricas em praticamente todas as áreas da vida humana, permitindo que
diversas tarefas pudessem ser realizadas de maneira rápida e eficaz em
comparação aos mesmos resultados obtidos por motores a combustão.

Para alimentar toda essa nova rede de consumidores e aparelhos que agora podem
aproveitar das mais diversas invenções que utilizam eletricidade, foi-se
necessário a construção de diversos geradores de energia. No Brasil, por
exemplo, optou-se pela geração hidroelétrica, que permite transformar a energia
cinética das águas em energia elétrica. Este processo é realizado com um gerador
síncrono e, associado ao transformador, permite a construção de uma ampla rede
de distribuição de energia elétrica. Essa rede alterou profundamente a forma
como o ser humano interage com o seu meio, já que ela transformou a cara das
cidades com os postes necessários para a transmissão.

A geração de energia elétrica também permitiu o uso de máquinas elétricas
pela indústria. Neste sentido, uma extensa gama de motores foi criada para
atender a diversas necessidades da economia. Vários motores de precisão foi
criados para poder construir novos produtos em uma escala muito maior à que era
permitida antes do século XIX, quando o trabalho manual ainda era a maior
força por trás da manufatura. Neste sentido, a utilização de máquinas mais
fortes e mais rápidas que o homem permitiu, por exemplo, a fabricação em série
dos automóveis, que também mudaram muito a vida do ser humano, não só por
substituirem os cavalos para várias atividades corriqueiras, mas também por
requerirem uma infraestrutura tão grande quanto a rede de transmissão elétrica
para funcionarem.

Os avanços das máquinas elétricas não ficaram presos somente à indústria.
Atualmente, diversas operações médicas só são possíveis graças à ciência e à
engenharia aplicadas na área. Desde pequenos equipamentos de laboratório até
a máquina de ressonância magnética usam motores elétricos para poder completar
operações anteriormente inviáveis para o ser humano sozinho. Outra indústria
que cresceu muito com a energia elétrica foi a extração mineral, tanto com
a mineração como a petroleira. Várias plataformas de extração de petróleo
ao redor do mundo usam motores e geradores elétricos constantemente para manter
suas operações, não somente de construção e manutenção de poços, mas na
infraestrutura necessária para que todo o processo seja concluído.

É inegável que todas essas áreas alteraram profundamente a vida do ser humano
nos últimos 200 anos, e tais avanços somente foram possíveis graças à
utilização da energia elétrica, principalmente aplicada à construção de
máquinas como o motor e o gerador.

\end{document}
