\documentclass[12pt, a4paper, twoside]{article}
\usepackage[utf8]{inputenc}
\usepackage[cm]{fullpage}
\usepackage{fancyhdr}
\usepackage{textcomp}
\usepackage{graphicx}
\usepackage{commath}
\usepackage[portuguese]{babel}

\begin{document}

\title{Qual a importância de um(a) engenheiro(a) ser capaz de prever, por meio
de equações matemáticas, o comportamento das máquinas elétricas?}
\author{Cristiano Silva Júnior: 13/0070629}
\date{\today}
\maketitle

Desde o princípio da história da Matemática, o ser humano a utiliza para tornar
seus esforços cada vez mais eficientes. A partir de poucos avanços na
trigonometria, por exemplo, os primeiros arquitetos foram capazes de projetar e
construir (com uma mão-de-obra gigantesca) as pirâmides do Egito. A simples
aplicação do que viria a se chamar por teorema de Pitágoras permitiu que os
construtores pudessem agir de maneira rápida e direta acerca de vários
problemas recorrentes na construção.

Com o passar do tempo, vários avanços na Matemática ocorreram e a Física
surgiu enquanto ciência. Sendo assim, o ser humano foi capaz de se tornar não
somente um observador passivo melhor, mas um pensador ativo, que é capaz de
imaginar uma situação e poder tomar decisões mesmo que nada tenha sido
construído ainda. Com isso, os projetos de engenharia se tornaram ainda mais
eficientes. Desde a definição das equações diferenciais por Newton no século
XVII até os dias de hoje, uma nova metodologia de trabalho surgiu e, com ela,
a engenharia pode crescer bastante com a utilização de equações que agora
podiam prever o resultado da operação das mais diversas máquinas.

Essa sólida fundação permitiu que físicos e engenheiros a superar os limites
que  definiam o que era fisicamente possível, até chegarmos ao nível que
estamos hoje, utilizando eletrons para realizar contas em nossos bolsos sem
nos darmos conta do tremendo esforço que é colocado nesse processo. Neste
sentido, tudo se deve à aplicação da Matemática para modelar o mundo real de
forma exata para prever fenômenos diversos.

Neste sentido, é essencial que os engenheiros da atualidade compreendam a
importância das ferramentas desenvolvidas e testadas ao longo de gerações de
profissionais da área e, mais importante ainda, que as estude e compreeenda os
princípios que permeiam todo esse esforço. Um entendimento profundo das
equações que regem as leis físicas, o contexto em que elas se aplicam e o
desenvolvimento de uma capacidade crítica para saber quando aplicar ou não
determinadas leis é essencial para que qualquer projeto de engenharia obtenha
os resultados desejados.

\end{document}
