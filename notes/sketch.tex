\documentclass[12pt, a4paper, twoside]{article}
\usepackage[utf8]{inputenc}
\usepackage[cm]{fullpage}
\usepackage{fancyhdr}
\usepackage{textcomp}
\usepackage{graphicx}
\usepackage{commath}
\usepackage[portuguese]{babel}

\begin{document}

\title{Esboço do projeto teórico de Conversão Eletromecânica de Energia}
\author{Cristiano Alves da Silva Júnior: 13/0070629}
\date{\today}
\maketitle

\section{Introdução}

Um problema relevante para todos os sistemas de saúde do planeta é o tamanho da fila para transplante de coração. Como é uma possibilidade de cirurgia cada vez mais comum, ela vem sendo bastante requisitada pelas organizações hospitalares. Contudo, sua dificuldade, tanto de acesso e como de execução, faz com que poucas pessoas possam realizá-la logo de imediato, tendo que esperar para poder receber um novo coração. Uma possível melhora nesta situação surge quando colocamos o coração artificial em jogo. Ele permite que as pessoas tenham um coração mecânico, que realiza as mesmas funções de um natural e extenderia a vida do paciente até um transplante completo. As limitações reais de um coração mecânico substituir completamente um coração natural são decorrentes do desgaste que os materiais utilizados na fabricação do dispositivo sofrem dentro do corpo humano, além de necessitarmos de uma bateria implantada dentro do sujeito para alimentar o coração. Tendo em vista o tempo que a bateria dura dentro do corpo humano, é atualmente mais viável esperar a bateria chegar até perto do fim do seu ciclo para transplantar um novo coração em vez de trocá-la.

Diante desta perspectiva, a motivação deste trabalho é dimensionar o consumo da bomba presente dentro do coração artificial e determinar a alimentação mínima necessária para mantê-la funcionando. Em vez de trocar a bateria, vamos supor que a alimentação vem de uma tatuagem transmissora de corrente elétrica. Sendo capaz de transmitir energia elétrica e conduzir calor, a pele humana pode servir como uma fonte de energia elétrica, e a transmissão desta energia seria feita por uma tinta especial. Esta energia pode ser utilizada para alimentar a bateria que viabiliza o funcionamento do coração, ou, mais ainda, alimentar o próprio coração diretamente.

Para entender o propósito deste projeto, temos que enteder como funcionam tatuagens condutivas. Como explicam os membros da Javey Research Group que desenvolveram a chamada E-Skin, o que temos não são tatuagens permanentes em si, mas sim uma pele artificial com circuito integrados capazes de interagir e se integrar com o o substrato humano. Desta forma, podemos realizar medidas eletrofisiológicas, interações elétricas com o corpo humano ou qualquer outra superfície que seja passível de se usar esta tecnologia, entre outras possibilidades. A ideia a ser explorada aqui é a inserção de células solares capazes de gerar tensão para ser guardada de alguma forma dentro desta e-skin para ser transmitida para o coração artificial. Como o dimensionamento da bomba, podemos decidir se este mecanismo imaginado é viável ou não.

\section{Metodologia}

O coração é um músculo cuja principal função é bombear sangue para as diversas partes do corpo. Para tanto, ele segue algumas "regras" para o sangue seja bem distribuído, tanto em questão de quantidade como de frequência. Na medicina, ele é conhecido por ser um músculo formado por células do tipo \textit{syncytium} (ou sincício em português), onde cada membrana nuclear cobre vários núcleos. Desta forma, uma excitação elétrica, por exemplo, se propaga rapidamente ao longo do órgão, haja visto a interconexão inerente de suas células. Isto é refletido em toda a estrutura elétromecânica do coração. Em termos estruturais, podemos dividir o coração em 4 partes: dois átrios e dois ventrículos. Os átrio e ventrículo do lado direito recebem sangue do corpo para bombear ao pulmão, enquanto os do lado esquerdo recebem sangue do coração para bombear para o corpo. Isto é necessário pois o sangue precisa estar rico em oxigênio, sendo o pulmão o órgão responsável por fazer este preparo químico. Desta forma, o coração precisa de um ritmo específico para trabalhar, e este ritmo pode ter sua velocidade aumentada ou diminuída dependendo de diversos estímulos externos.

Os átrios são responsáveis por receber o sangue que chega ao coração. Eles funcionam como bombas mais fracas, e simplesmente bombeiam sangue ao ventrículo que, por sua vez, já funciona como uma bomba mais forte, que deve ser capaz de fazer pressão suficiente para empurrar o sangue distribuído tanto pelo pulmão como pelo resto corpo. Desta forma, é fácil ver que a parte direita do coração exerce menos pressão do que a parte esquerda, já que, como a circulação do pulmão é independente da do resto do corpo, ela requer que menos sangue passe por ela. Também é notável que, do ponto de vista de sistemas dinâmicos, o pulmão é um gargalo, pois ele não é capaz de processar todo o sangue do corpo de uma só vez. Isto é perceptível em todos aqueles que praticam exercício físico: a sensação de estar sem ar após a realização de exercícios pesados provém da incapacidade do coração de bombear sangue o suficiente pelo pulmão para distribuir oxigênio por todo o corpo. Sendo assim, é necessária a distinção entre circulação pulmonar e circulação sistêmica (ou periférica), haja visto suas peculiaridades inatas.

Após sair do coração em direção à circulação sistêmica, o sangue passa pelas artérias (sendo a mais importante a artéria Aorta, a primeira artéria do coração e a que aguenta mais pressão), as vias de alta pressão. As artérias se dividem em arteríolas, que por sua vez se dividem em capilares. As vias capilares servem para distribuir o sangue dentro dos órgãos, para que haja a troca de nutrientes e substâncias necessárias para a manutenção do corpo humano. Após a realização das reações, o sangue volta à circulação das capilares pelas vênulas, que se juntam em veias, capazes de aguentar pressões maiores, permitindo um retorno mais rápido ao coração. O coração deve então bombear este sangue pelos pulmões, onde um caminho similar é percorrido: o sangue sai pela chamada \textit{vena cava} superior, passa pelos pulmões através da circulação capilar, e retorna ao coração pela chamada artéria \textit{cava}.

Com isto em mente, o objetivo deste trabalho é dimensionar duas bombas capazes de bombear sangue intermitentemente, tanto para o pulmão como para a circulação sistêmica. Em termos mecânicos, podemos dimensionar estas bombas se baseando na pressão exercida sobre as artéria Aorta e \textit{vena cava} superior. Segundo Guyton et al. (1970s), que determinaram o chamado débito cardíaco (isto é, a diferença entre o tanto de sangue que chega ao coração e o quanto que sai), eles chegaram à conclusão de que o coração flui pela Aorta com uma velocidade de $F=33cm/s$ sob condições de repouso. Este valor de velocidade chega a se multiplicar por um fator de 8 quando o coração está sob exigência de mais nutrientes nas vias capilares. Voltando ao estado de repouso, os pesquisadores também foram capazes de medir a pressão exercida sob as paredes da Aorta, chegando a um máximo médio de $P_s=120mmHg$ para pressões sistólicas (exercidas para um único batimento) e $P_d=80mmHg$ para pressões diastólicas (o mínimo exercido entre dois batimentos). Estas pressões não mudam se o sujeito está sob repouso ou durante alguma atividade física.

A saber, a prova de que o pulmão é um gargalo neste sistema vem das necessidades nutricionais impostas pelas vias capilares, que demandam 30 vezes o fluxo de sangue do coração quando em repouso. Contudo, o coração é mecanicamente incapaz de fornecere um fluxo maior do que no máximo 8 vezes (como explicado anteriormente).

\section{Resultados}

Para este projeto, vamos considerar que o coração é formado por duas bombas, uma para circulação pulmonar e outra para circulação sistêmica. Sabendo que a circulação pulmonar demanda menos pressão que a periférica, podemos simplificar este problema levando em conta, na verdadem a potência necessária para alimentar duas artérias Aorta sob condições limite, no caso, quando o sujeito está em atividade física severa e estamos aplicando pressão sistólica sobre a artéria, requerindo assim 8 vezes o fluxo de sangue necessário para um indivíduo em repouso. Sendo assim, a potência necessária será, convertendo os valores para o \textit{SI},
$$ P = F \cdot P_s = 42236.53 W $$

A partir daqui, eu vou escolher um motor para a atividade e checar quanta potência elétrica deve ser necesária.

\section{Conclusão}

Aqui, eu deverei falar sobre a possibilidade de se aplicar os resultados obtidos em uma aplicação real.

\section{Bibliografia}

\begin{itemize}
    \item GUYTON, Arthur. HALL, John. "Textbook of Medical Physiology". Elsevier, 2016, 13th edition.
    \item url<http://nano.eecs.berkeley.edu/research/eskin.html>. Acesso em 26 de Setembro.
\end{itemize}

\end{document}
