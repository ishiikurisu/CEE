\documentclass[12pt, a4paper, twoside]{article}
\usepackage[utf8]{inputenc}
\usepackage[cm]{fullpage}
\usepackage{fancyhdr}
\usepackage{textcomp}
\usepackage{graphicx}
\usepackage{commath}
\usepackage[portuguese]{babel}

\begin{document}

\title{Esboço do projeto teórico de Conversão Eletromecânica de Energia}
\author{Cristiano Alves da Silva Júnior: 13/0070629}
\date{\today}
\maketitle

\section{Introdução}

Um problema relevante para todos os sistemas de saúde do planeta é o tamanho da fila para transplante de coração. Como é uma possibilidade de cirurgia cada vez mais comum, ela vem sendo bastante requisitada pelas organizações hospitalares. Contudo, sua dificuldade, tanto de acesso e como de execução, faz com que poucas pessoas possam realizá-la logo de imediato, tendo que esperar para poder receber um novo coração. Uma possível melhora nesta situação surge quando colocamos o coração artificial em jogo. Ele permite que as pessoas tenham um coração mecânico, que realiza as mesmas funções de um natural e extenderia a vida do paciente até um transplante completo. As limitações reais de um coração mecânico substituir completamente um coração natural são decorrentes do desgaste que os materiais utilizados na fabricação do dispositivo sofrem dentro do corpo humano, além de necessitarmos de uma bateria implantada dentro do sujeito para alimentar o coração. Tendo em vista o tempo que a bateria dura dentro do corpo humano, é atualmente mais viável esperar a bateria chegar até perto do fim do seu ciclo para transplantar um novo coração em vez de trocá-la.

Diante desta perspectiva, a motivação deste trabalho é dimensionar o consumo da bomba presente dentro do coração artificial e determinar a alimentação mínima necessária para mantê-la funcionando. Em vez de trocar a bateria, vamos supor que a alimentação vem de uma tatuagem transmissora de corrente elétrica. Sendo capaz de transmitir energia elétrica e conduzir calor, a pele humana pode servir como uma fonte de energia elétrica, e a transmissão desta energia seria feita por uma tinta especial. Esta energia pode ser utilizada para alimentar a bateria que viabiliza o funcionamento do coração, ou, mais ainda, alimentar o próprio coração diretamente.

Para entender o propósito deste projeto, temos que enteder como funcionam tatuagens condutivas. Como explicam os membros da Javey Research Group que desenvolveram a chamada E-Skin, o que temos não são tatuagens permanentes em si, mas sim uma pele artificial com circuito integrados capazes de interagir e se integrar com o o substrato humano. Desta forma, podemos realizar medidas eletrofisiológicas, interações elétricas com o corpo humano ou qualquer outra superfície que seja passível de se usar esta tecnologia, entre outras possibilidades. A ideia a ser explorada aqui é a inserção de células solares capazes de gerar tensão para ser guardada de alguma forma dentro desta e-skin para ser transmitida para o coração artificial. Como o dimensionamento da bomba, podemos decidir se este mecanismo imaginado é viável ou não.

\section{Metodologia}

Aqui, eu deverei falar sobre os fundamentos que regem as contas que eu farei mais tarde.

Falar sobre o fluxo de sangue pelo coração, indicando quanto sangue passa pelo coração e qual a potência mecânica necessária.

Falar sobre a fragilidade elétrica do coração.

\section{Resultados}

Aqui, eu deverei dimensionar o motor da bomba do coração, levando em conta os parâmetros descritos na introdução.

\section{Conclusão}

Aqui, eu deverei falar sobre a possibilidade de se aplicar os resultados obtidos em uma aplicação real.

\section{Bibliografia}

Basicamente, alguns artigos do IEEE. E este site: http://nano.eecs.berkeley.edu/research/eskin.html.

\end{document}
