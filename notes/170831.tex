\documentclass[12pt, a4paper, twoside]{article}
\usepackage[utf8]{inputenc}
\usepackage[cm]{fullpage}
\usepackage{fancyhdr}
\usepackage{textcomp}
\usepackage{graphicx}
\usepackage{commath}
\usepackage[portuguese]{babel}

\begin{document}

\section{Aula do dia 31 de Agosto de 2017}


\subsection{Exemplo 1}

Seja um trafo de $15kVA$ com especificações $2300/230V$. Preencha a tabela, calcule rendimento e regulação para carga nomimal com $fp = 0,8$ atrasado.

\begin{center}
    \begin{tabular}{ | c | c | c | }
        \hline
         & Ensaio Vazio BT & Ensaio Curto Circuito AT \\ \hline
        V   & ? & ? \\ \hline
        I   & ? & ? \\ \hline
        P   & ? & ? \\ \hline
    \end{tabular}
\end{center}

$V$ está em volts; $I$, em amperes; e $P$, em watts.

\subsection{Resolução do exemplo 1}

Utilizando os valores nominais do trafo, podemos esperar os seguintes valores:

\begin{center}
    \begin{tabular}{ | c | c | c | }
        \hline
        ... & Ensaio Vazio BT & Ensaio Curto Circuito AT \\ \hline
        V   & 230             & 47                       \\ \hline
        I   & 2,1             & 6,0                      \\ \hline
        P   & 50              & 160                      \\ \hline
    \end{tabular}
\end{center}

No ensaio em vazio, vamos aplicar a menor tensão nominal no lado de baixa tensão. Para descobrir o $R_c$ (a resistência associada ao ferro) e o $X_m$ (a indutância relacionada à magnetização do ferro), podemos usar a lei de Ohm:

$$ P = \frac{V^2}{R_c} \Rightarrow R_c = \frac{V^2}{P} = 1058,0000 \Omega $$
$$ Z = \frac{V}{I} = 109,5238 $$
$$ (\frac{1}{Z})^2 = (\frac{1}{R_c})^2 + (\frac{1}{X_m})^2 \Rightarrow 10,1154 \Omega $$

Com estes parâmetros do trafo, podemos calcular o seu fator de potência:

$$ S = VI = 230 * 2,1 = 483,0000 VA $$
$$ P = 50W $$
$$ fp = \frac{P}{S} = 0,1035 $$

No ensaio em curto circuito, vamos a aplicar a corrente nominal no lado de alta tensão:

$$ P = R_{eq} * I^2 \Rightarrow R_{eq} = \frac{P}{I^2} = 4,4444 \Omega $$
$$ Z = \frac{V}{I} = 7,8333 \Omega $$
$$ Z^2 = R_{eq}^2 + X_{eq}^2 \Rightarrow X_{eq} = 6,4504 \Omega $$

O fator de potência neste lado do circuito é:

$$ S = VI = 46 * 7 = ? VA $$
$$ P = 160W$$
$$ fp = \frac{P}{S} = 0,56 \Rightarrow \phi = 56^o $$

Para proceder com as contas, temos que referenciar o transformador ou à alta tensão ou à baixa tensão. Por exemplo, em baixa tensão, $R_c = 1058 \Omega$. Já em alta tensão, $R_c = R_{c BT} a^2 = ? $ onde
$$ a = \frac{AT}{BT} = \frac{2300}{230} $$
Se, em AT, $ R = 4,444 $, então $R_{BT} = \frac{R_{AT}}{a^2} = 0,0444 \Omega $

Para calcular o rendimento, vamos colocar referenciar o transformador para a alta tensão e aplicar a alta tensão nominal em uma carga que consuma $ S = 15000VA $.

Para calcular a regulação de tensão (RT), utilizamos a definição. A $RT$ mede o quanto a tensão nos casos extremos de aplicação, a chamada dispersão. Ela deve ser pequena (no máximo 3\%) para evitar que haja problemas em instalações elétricas.

\end{document}
