\documentclass[12pt, a4paper, twoside]{article}
\usepackage[utf8]{inputenc}
\usepackage[cm]{fullpage}
\usepackage{fancyhdr}
\usepackage{textcomp}
\usepackage{graphicx}
\usepackage{commath}
\usepackage[portuguese]{babel}

\begin{document}

\title{Esboço do projeto teórico de Conversão Eletromecânica de Energia}
\author{Cristiano Alves da Silva Júnior: 13/0070629}
\date{\today}
\maketitle

\section{Introdução}

Aqui, eu deverei explicar porque este problema é relevante e qual a proposta da solução.

\section{Metodologia}

Aqui, eu deverei falar sobre os fundamentos que regem as contas que eu farei mais tarde.

% TODO Falar sobre o fluxo de sangue pelo coração, indicando quanto sangue passa pelo coração e qual a potência mecânica necessária.

% TODO Falar sobre a fragilidade elétrica do coração.

\section{Resultados}

Aqui, eu deverei dimensionar o motor da bomba do coração, levando em conta os parâmetros descritos na introdução.

\section{Conclusão}

Aqui, eu deverei falar sobre a possibilidade de se aplicar os resultados obtidos em uma aplicação real.

\section{Bibliografia}

Basicamente, os artigos do IEEE.

\end{document}
